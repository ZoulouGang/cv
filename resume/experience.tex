%-------------------------------------------------------------------------------
%	SECTION TITLE
%-------------------------------------------------------------------------------
\vspace{-2.0mm}

\cvsection{Expériences}

\cvsubsection{Ingénieur logiciel chez Vinci Autoroutes}
%-------------------------------------------------------------------------------
%	CONTENT
%-------------------------------------------------------------------------------
\begin{cventries}

%---------------------------------------------------------
  \cventry
    {Suite applicative de télé assistance, de télé exploitation et de contrôle en temps réel des gares de péage} % Job title
    {Application de Téléopération (ADTO)} % Organization
    {Rueil-Malmaison} % Location
    {Depuis Janvier 2012} % Date(s)
    {
      \begin{cvitems} % Description(s) of tasks/responsibilities
        \item {Conception en collaboration avec le client, définition des différentes pages de la solution, rédaction des spécifications}
        \item {Développement des applications au sein d'une équipe de 5 ingénieurs}
        \item {Présentation aux équipes de validation et au client final}
        \item {Vérification du bon fonctionnement de la solution lors de son déploiement progressif}
        \item {Visites sur les sites de téléopération afin de constater les problèmes rencontrés lors de la montée en charge de l'outil}
        \item {Analyse, avec la production informatique, des problèmes de performance lors de la montée en charge}
        \item {Implémentation d'optimisations: ajout d'un système de cache, optimisation de requêtes}
        \item {Localisation des points de contention et résolution de problèmes de stabilité au niveau du navigateur Web}
        \item {Intégration de l'interface entre Milestone et ADTO afin de piloter les caméras suite au changement du système vidéo de Cofiroute}
      \end{cvitems}
    } 
\end{cventries}

~~~~\\

\begin{cventries}

%---------------------------------------------------------
  \cventry
    {Permet de réduire les erreurs de tarification et d'identifier les transactions frauduleuses} % Job title
    {Outils de Contrôle et de Correction Différé des Transactions (OCCDT)} % Organization
    {Rueil-Malmaison} % Location
    {Janvier 2013 - 2017} % Date(s)
    {
      \begin{cvitems} % Description(s) of tasks/responsibilities
        \item {Reprise de l'application à la suite de la première mise en production}
        \item {Ajout du volet fraude, et liaison avec l'outil de recouvrement qui prend le relais une fois la fraude confirmée}
        \item {Intégration de la vidéo à la place des photos dans l'outil de qualification de la fraude}
        \item {Évolution de la mise à disposition des photos de transaction afin de les remonter en temps réel}
        \item {Ajout d'un module de correction automatique des erreurs de trajet}
        \item {Optimisation de l'interface en relation avec les utilisateurs et leur encadrement}
      \end{cvitems}
    } 

%---------------------------------------------------------
  \cventry
    {Outils regroupant le téléphone, la vidéo et la main courante dans une page afin d'assister les clients en voie} % Job title
    {Outil de TéléOpération (OTO 2.0)} % Organization
    {Sèvres} % Location
    {Juin 2010 - Jan. 2012} % Date(s)
    {
      \begin{cvitems} % Description(s) of tasks/responsibilities
        \item {Développeur principal dans une équipe de deux personnes}
        \item {Interlocuteur technique lors des phases de validation, de mise en production et de suivi en production}
        \item {Maquettage en accord avec un ergonome des deux principaux écrans}
      \end{cvitems}
    }

%---------------------------------------------------------
  \cventry
    {Application de configuration du système d’interphonie de Cofiroute, équipe composée d'un ingénieur système et deux développeurs} % Job title
    {Oudin} % Organization
    {Sèvres} % Location
    {Mai 2009 - Juin 2010} % Date(s)
    {
      \begin{cvitems} % Description(s) of tasks/responsibilities
        \item {Développement des interfaces web, la fonction principale étant de définir les postes appelés par chaque interphone}
        \item {Génération des fichiers de configuration Asterisk à partir des liens effectués par l'utilisateur}
        \item {Mise en place du provisioning des postes SIP, facilitant leur mise en place et leur remplacement}
        \item {Participation à l'étude des différents modèles de postes téléphoniques candidats}
      \end{cvitems}
    }

%---------------------------------------------------------



%---------------------------------------------------------
\end{cventries}

\cvsubsection{Consultant chez Airial Conseil}


\begin{cventries}
  \cventry
    {Refonte du logiciel de gestion des ressources humaines du ministère de l'agriculture} % Job title
    {Système RH du Ministère de l'agriculture (Agorha)} % Organization
    {Paris 6} % Location
    {2007 - 2009} % Date(s)
    {
      \begin{cvitems} % Description(s) of tasks/responsibilities
        \item {Développement dans une équipe de 20 personnes, répartie entre développement et spécification}
        \item {Aide à la spécification à partir du code source de l'application historique}
        \item {Formation et suivi des développeurs novices intégrés à l'équipe}
        \item {Rédaction de demandes de modification du Framework interne à l'équipe du Ministère responsable des outils}
      \end{cvitems}
    }
\end{cventries}

\begin{cventries}
  \cventry
    {Projet s'appuyant sur les technologies du web sémantique pour permettre de simplifier la communication entre les services gouvernementaux} % Job title
    {Projet Européen Terregov} % Organization
    {Puteaux, La défense} % Location
    {2006 - 2007} % Date(s)
    {
      \begin{cvitems} % Description(s) of tasks/responsibilities
        \item {Responsable du module d'annotation sémantique de web service}
        \item {Participation aux réunions techniques avec quinze partenaires du consortium répartis dans dix pays}
      \end{cvitems}
    }

\end{cventries}
